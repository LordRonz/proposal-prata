\chapter*{ABSTRAK}

\begin{center}
  \textbf{SISTEM BERBAGI DATA BERBASIS \emph{BLOCKCHAIN} UNTUK \emph{AUDIO PLAYER}
  DI \emph{METAVERSE}}
\end{center}
\addcontentsline{toc}{chapter}{ABSTRAK}
% Menyembunyikan nomor halaman
\thispagestyle{empty}

\begin{flushleft}
  \setlength{\tabcolsep}{0pt}
  \bfseries
  \begin{tabular}{ll@{\hspace{6pt}}l}
  Nama Mahasiswa / NRP&:& Aaron Christopher Tanhar / 07211940000055\\
  Departemen&:& Teknik Komputer FTEIC - ITS\\
  Dosen Pembimbing&:& 1. Mochamad Hariadi, S.T., M.Sc., Ph.D.\\
  & & 2. Dr. Surya Sumpeno, S.T., M.Sc.\\
  \end{tabular}
  \vspace{4ex}
\end{flushleft}
\textbf{Abstrak}

% Isi Abstrak
Metaverse merupakan sebuah seperangkat ruang virtual, tempat seseorang dapat membuat dan menjelajah 
dengan pengguna internet lainnya yang tidak berada pada ruang fisik yang sama dengan orang tersebut.
Pengaplikasiannya kerap kali menggunakan \emph{blockchain} sebagai solusi \emph{decentralized}. Metaverse sendiri
tentunya memerlukan adanya output audio, yang diwujudkan oleh Unreal Engine 5 dengan fitur metasoundnya.

Penggabungan keduanya dapat dicapai dengan menggunakan \emph{smart contract}. Maka di penelitian ini akan dibuat sistem yang dapat mewujudkan hal tersebut
dengan bantuan \emph{Ethereum Smart Contract}, NFT untuk menyimpan metadata, dan web3.storage IPFS untuk menyimpan file sumber audio untuk metasound.

Dengan adanya sistem terdesentralisasi tersebut juga diharapkan terwujudnya \emph{interoperability} agar metaverse ini dapat digunakan dan berkomunikasi dengan platform
blockchain lainnya.

\vspace{2ex}
\noindent
\textbf{Kata Kunci: \emph{Metaverse, Audio, Blockchain, Ethereum, NFT}}