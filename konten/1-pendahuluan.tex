\section{PENDAHULUAN}

\subsection{Latar Belakang}

% Ubah paragraf-paragraf berikut sesuai dengan latar belakang dari tugas akhir
Musik adalah salah satu ruang terbesar di Industri Hiburan. Selama bertahun-tahun, ada banyak unduhan yang dilakukan di industri musik dalam beragam format digital
dan ekstensi (mis., mp4, mp3, dll.). Namun, sebagian besar unduhan yang disebutkan di atas diperoleh secara ilegal namun diindeks
atau dicatat dalam perpustakaan digital unduhan global. Karena aspek unduhan ilegal, pembuat konten kehilangan banyak pendapatan
dan menimbulkan biaya yang mengarah pada masalah hak cipta.

Sejak diperkenalkan pada tahun 2008, teknologi blockchain telah dipuji sebagai salah satu yang bisa
merevolusi banyak industri yang berbeda. Industri musik mengalami beberapa perubahan dalam 20 tahun terakhir sebagai konsekuensi dari fenomena seperti pembajakan musik, musik digital dan streaming musik. Banyak
menganggap teknologi blockchain sebagai solusi untuk banyak masalah yang dihadapi industri musik.
Namun, sementara teknologi telah ada selama lebih dari 10 tahun, dan terlepas dari antusiasme
sarjana dan ahli, sedikit yang telah dilakukan untuk benar-benar menerapkan blockchain di industri,
terutama oleh pemain terbesarnya. Pertanyaan yang dihasilkan adalah bagaimana teknologi dapat mengubah
industri musik.

Diperlukan sebuah sistem yang bisa berbagi record data dengan integritas yang tinggi. Kebanyakan sistem yang ada
menggunakan sistem terpusat yang masih mempunyai kelemahan
dalam integritas sebuah data. Maka dari itu, kami memberikan
solusi dengan menggunakan sistem blockchain. Blockchain mampu
menyediakan integritas yang tinggi dengan sistemnya yang bersifat
desentralisasi, dapat diakses secara global oleh siapa saja. 
Di metaverse, NFT merupakan konsep penting dalam ekosistem metaverse, 
yang memungkinkan orang untuk memiliki benda virtual dalam bentuk seperti mobil, kapal, atau bahkan aksesori dan lukisan maupun musik, semuanya dimungkinkan melalui NFT. 
Non Fungible Token (NFT) adalah aset kriptografi pada blockchain dengan kode identifikasi unik dan memiliki metadata yang membedakannya satu sama lain.

\subsection{Rumusan Masalah}

% Ubah paragraf berikut sesuai dengan rumusan masalah dari tugas akhir
Berdasarkan dari latar belakang diatas , maka adapun permasalahan yang dapat diambil adalah diperlukan sistem blockchain yang dapat digunakan sebagai basis dari NFT untuk melakukan file sharing untuk music player pada metaverse. 

\subsection{Penelitian Terkait}

% Ubah paragraf berikut sesuai dengan penelitian lain yang terkait dengan tugas akhir
\subsubsection{Pengembangan Sistem Keamanan Berbagi Data PACS Berbasis Blockchain}
Penulis pada \citep{DitoPrabowo2020} membuat sebuah sistem berbagi record data PACS pasien dalam penelitian ini
telah berhasil dalam mengintegrasikan record data pasien antar ru-
mah sakit juga tambahan keamanan dengan pemberian akses untuk
berbagi, rumah sakit yang mendapat akses token dan secret key da-
pat membaca maupun menulis data pasien sesuai akses token yang
diberikan pasien. Bagi yang tidak mendapatkan akses tidak akan
bisa membaca maupun menulis data pasien, dengan ini keamanan
dan privasi pasien terjaga penuh. Kendali atas datanya tetap pa-
sien yang menentukan. Kejadian data breach juga sulit dilakukan
pada sistem aplikasi yang bersifat desentralisasi dan menggunakan
platform blockchain apalagi data juga dienkripsi.
Terkait dengan scalability, sistem mampu menangani berba-
gai size file yang beragam mulai dari 1Mb sampai dengan 64 Mb,
dan dari dilakukannya beberapa pengujian didapatkan hasil ekseku-
si waktu yang mendekati sama, dari ini bisa diketahui bahwa IPFS
stabil, juga bisa lebih cepat apabila file sudah pernah unggah sebe-
lumnya (hash file sama).
Pada fitur aplikasi, semakin banyak data yang di buat block
akan semakin banyak waktu yang diperlukan untuk mining. Ter-
bukti pada feature panambahan data pasien didapatkan rata-rata
waktu mining 0,51 detik, lebih lama dari feature yang lain.

\subsubsection{Securing music sharing platforms: A Blockchain-Based Approach}
Penulis pada paper \citep{adjei2021securing} mengajukan sebuah platform berbagi musik yang berbasis pada blockchain dan arsitektur IPFS. 
Proposisi dari penulis adalah menghilangkan dan meminimalkan pembagian musik ilegal dari pembuat konten di seluruh
internet menggunakan teknologi blockchain yang juga memfasilitasi pemeriksaan duplikat metadata di
Internet. Jaringan berbagi file ini dibangun berdasarkan blockchain Ethereum yang menggunakan
cara mekanisme konsensus untuk mencapai tujuan yang dinyatakan dari \emph{smart contract} dengan cara yang cepat dan aman.
Simulasi penulis menyajikan berbagai langkah yang diperlukan untuk memvisualisasikan pengoperasian sistem yang diusulkan. Penulis memperkenalkan fitur pendaftaran dan kontrol akses yang ditambahkan ke protokol IPFS untuk memastikan bahwa
file musik memiliki hak cipta. \emph{Smart contract} memastikan bahwa persyaratan yang harus dipenuhi sebelumnya
mengakses file musik diberlakukan. Keadaan jaringan yang tahan-rusak ditunjukkan sedemikian rupa sehingga
setiap node yang berpartisipasi memastikan bahwa catatan yang disimpan pada file musik yang diunduh sesuai dengan yang diharapkan
pendapatan. Akuntabilitas pendapatan yang efektif dicapai dengan sistem yang diusulkan.

\subsection{Tujuan Penelitian}

% Ubah paragraf berikut sesuai dengan tujuan penelitian dari tugas akhir
Penelitian ini memiliki tujuan untuk membuat sistem yang
dapat melakukan sharing data untuk musical player yang ada di Metaverse menggunakan platfrom blockchain, agar pengguna metaverse dapat menggunakan platform music dan diintegrasikan dengan sistem rekomendasi musik.