\chapter{PENDAHULUAN}

\section{Latar Belakang}

% Ubah paragraf-paragraf berikut sesuai dengan latar belakang dari tugas akhir
Munculnya teknologi \emph{virtual reality (VR)} dan pengembangan platform metaverse telah mengarah pada terciptanya
pengalaman online yang imersif di mana pengguna dapat berinteraksi satu sama lain dan terlibat dalam berbagai aktivitas, apalagi setelah  Mark Zuckerberg
membeli \emph{oculus} \parencite{luckerson2014facebook}.
Salah satu aktivitas tersebut adalah memainkan alat musik sebagai karakter virtual, yang dapat dicapai melalui penggunaan
teknologi \emph{motion capture} dan simulasi instrumen virtual.

Namun, sistem saat ini untuk berbagi dan mengakses data musik di metaverse bersifat terpusat dan sering mengandalkan teknologi
hak cipta, yang dapat membatasi fleksibilitas dan interoperabilitas pengalaman musik. Pendekatan yang terdesentralisasi dan terbuka
untuk berbagi data dapat memungkinkan ekosistem musik yang lebih beragam dan interaktif di dalam \emph{metaverse}.

Kemudian untuk layanan-layanan pemutar musik juga memiliki beberapa implementasi system \emph{centralized}. Contohnya spotify, soundcloud, dan lain-lain.
Soundcloud sendiri memiliki kelemahan pada sisi kepemilikan atau \emph{copyright}. Sistem \emph{decentralized} merupakan salah satu solusi dari hal ini, dikarenakan
sifatnya yang \emph{immutable}. Salah satu contohnya adalah \emph{Audius}. Audius sendiri menggunakan sistem \emph{decentralized} dan source musiknya disimpan di IPFS.
Akan tetapi untuk menggunakan Audius harus menggunakan \emph{embedded iframe} yang artinya hanya support untuk penggunaan di \emph{Web Application} saja.

Adapula antarmuka dari audio itu sendiri adalah metaverse yang salah satu contohnya adalah Unreal Engine 5. Salah satu fitur dari Unreal Engine 5 yaitu Metasound yang dapat melakukan pengolahan audio dengan mudah bagi para pengembang. Unreal Engine 5 juga
memiliki fitur yakni API \emph{blockchain} sehingga dapat mengeksekusi \emph{smart contract}. Fitur ini dinamakan Web3.UE. Web3.Unreal adalah plugin
\emph{open source} yang dibuat untuk pengembang game dan komunitas untuk membantu mereka yang bekerja dengan Unreal Engine untuk mengintegrasikan
fungsionalitas \emph{blockchain} ke dalam game mereka. Kedua hal ini dapat dikombinasikan dan digunakan untuk
pengembangan metaverse.

Metaverse telah digambarkan sebagai iterasi baru dari internet yang menggunakan headset VR, teknologi \emph{blockchain},
dan avatar dalam integrasi baru dunia fisik dan virtual. \parencite{DWIVEDI2022102542}. Dalam proposal penelitian ini, saya mengusulkan
untuk merancang dan mengimplementasikan sistem berbagi data berbasis \emph{blockchain}
untuk \emph{musical player} di \emph{metaverse} dengan \emph{NFT}. Sistem ini akan menggunakan smart contract dan solusi penyimpanan terdesentralisasi
untuk memungkinkan pengguna berbagi dan mengakses data musik dari metaverse dengan cara yang aman dan transparan.

\section{Rumusan Masalah}

% Ubah paragraf berikut sesuai dengan rumusan masalah dari tugas akhir
Di dunia \emph{virtual metaverse}, kebutuhan akan sistem yang aman dan efisien untuk berbagi data terkait musik semakin meningkat.
Saat ini, data ini sering disimpan dalam database terpusat yang rentan terhadap pelanggaran keamanan dan
penyensoran, dan dapat menyulitkan pengguna untuk mengakses dan mengontrol data mereka sendiri. Dibutuhkan juga sistem \emph{blockchain} dengan interoperabilitas sehingga
pengembang metaverse dapat mengembangkan sistem yang dapat berkomunikasi dengan \emph{blockchain} yang lainnya. Karena belum ada sistem yang ada untuk melakukan hal serupa layaknya audius dan spotify di metaverse,
maka penelitian ini juga mencetuskan hal yang belum ada pada saat proposal ini diajukan.

\section{Batasan Masalah}

% Ubah paragraf berikut sesuai dengan rumusan masalah dari tugas akhir
Untuk memfokuskan permasalahan yang diangkat maka dilakukan pembatasan ma-
salah. Batasan - batasan masalah tersebut diantaranya:
\begin{enumerate}
  \item Jenis \emph{Blockchain} yang digunakan adalah Ethereum.
  \item Target pengguna adalah pengguna metaverse.
  \item Target platform musik adalah Metasounds di Unreal Engine 5.
  \item Data berisi JSON yang adalah metadata yang diperuntukkan untuk metasound.
  \item File audio disimpan di IPFS
\end{enumerate}

\section{Tujuan Penelitian}

% Ubah paragraf berikut sesuai dengan tujuan penelitian dari tugas akhir
Penelitian ini memiliki tujuan untuk membuat sistem yang
dapat melakukan sharing data untuk audio player yang ada
di Metaverse menggunakan platform \emph{blockchain}, agar
pengguna metaverse dapat menggunakan platform music
yang diintegrasikan dengan metasound. Penelitian ini juga bertujuan mengembangkan metaverse yang 
bersinergi dengan web3.0, dan diharapkan terjadinya interoperability sehingga tidak hanya dapat digunakan
dengan Ethereum saja.

\section{Manfaat}
Manfaat yang bisa didapat dari hasil tugas akhir ini adalah
memajukan pengembangan metaverse pada bidang audio dan sistem berbagi data \emph{blockchain}
terkhususnya bagi para pengembang Web3.
