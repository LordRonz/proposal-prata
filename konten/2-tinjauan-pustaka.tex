\section{TINJAUAN PUSTAKA}

% Ubah konten-konten berikut sesuai dengan isi dari tinjauan pustaka

\subsection{Jaringan Terdistribusi}

Jaringan Terdistribusi adalah sebuah jaringan dimana setiap
node dalam cluster tersebut saling terhubung satu sama lain. Ja-
ringan desentralisasi mendistribusikan data, informasi, dan memp-
roses beban kerja di seluruh komputer yang berpartisipasi dalam
jaringan. Hal ini memungkinkan untuk toleransi kesalahan yang
lebih besar sebagai komponen kunci dari jaringan di distribusikan
di beberapa mesin, jika salah satu jaringan (hardware) down, maka
jaringan secara keseluruhan terus berfungsi.
Jaringan terdesentralisasi memiliki keuntungan lebih aman da-
ripada jaringan terpusat, tidak ada satu server pusat yang ditargetk-
an oleh penyerang. Jika seseorang ingin menyerang jaringan, mereka
perlu untuk menyerang beberapa node dalam jaringan. Karena ti-
dak ada kekuasaan dalam jaringan, pengguna bisa lebih memperca-
yai sistem, namun juga karena tidak ada kekuasaan dalam jaringan,
apabila ada masalah dalam jaringan tidak ada otoritas pusat yang
membantu menyelesaikan masalah hal ini yang membedakan anta-
ra jaringan terpusat dan jaringan terdistribusi.

\subsection{Metaverse}

Istilah metaverse pertama kali digunakan oleh Neal Stephenson di novelnya \emph{Snow Crash} untuk
mendeskripsikan dunia virtual dimana karakter protagonisnya, \emph{Hiro Protagonist}
, bersosialisasi, belanja, dan mengalahkan musuh-musuh dunia nyata melalui avatarnya. Ini bukanlah konsep yang baru
. Faktanya, film garapan Steven Spielberg pada tahun 2018 yang berjudul \emph{Ready Player One} juga mengambil konsep yang serupa dan
menggunakannya dengan sangat ciamik. Definisi teknis dari Metaverse itu sendiri adalah sebuah ruang virtual dimana orang-orang didalamnya bisa berinteraksi 
dengan lingkungan yang dibuat oleh komputer dan pemain lain.

\subsection{ERC-721}
ERC-721 merupakan antarmuka standar untuk membuat \emph{Non Fungible Token}, atau NFT.
Use case untuk NFT adalah kepemilikan asset digital dan transaksinya hingga pengiriman ke 
\emph{crypto wallet} pihak ketiga.

\subsection{Solidity}
Solidity adalah adalah bahasa tingkat tinggi berbasis objek untuk mengimplementasikan
\emph{smart contract}. Solidity ini dijalankan didalam Ethereum Virtual Machine (EVM).
Kontrak di Solidity mirip dengan class dalam bahasa object-oriented. Setiap kontrak dapat berisi pernyataan tentang :ref:`structure-state-variables`, :ref:`structure-functions`, :ref:`structure-function-modifiers`, :ref:`structure-events`, :ref:`structure-errors`, :ref:`structure-struct-types` dan :ref:`structure-enum-types`. Lebih jauh, kontrak dapat mewarisi dari kontrak lain.

Ada juga jenis kontrak khusus yang disebut :ref:`libraries<libraries>` dan :ref:`interfaces<interfaces>`.

Bagian tentang :ref:`kontrak<contracts>` berisi lebih banyak detail daripada bagian ini, yang berfungsi untuk memberikan gambaran singkat.

\subsection{Blockchain}

Blockchain adalah database terdistribusi yang digunakan un-
tuk menangani record data yang terus bertambah, record data ini
disebut dengan block. Setiap block memiliki penanda waktu dan ko-
de unik yang terhubung dengan block sebelumnya, sehingga masing
masing block tersebut saling terhubung satu sama lainnya dan ti-
dak bisa untuk diubah. Blockchain biasanya dikelola oleh jaringan
peer-to-peer yang secara kolektif mengikuti protokol untuk memva-
lidasi block baru. Jika terdapat perintah penambahan block baru,
maka setiap node pada jaringan peer-to-peer tersebut akan terlebih
dahulu memvalidasi block dan kemudian seluruh node akan mem-
perbarui record data miliknya [18]. Blockchain memiliki 3 type [19]
yaitu Public Blockchain yang dikembangkan secara bersama-sama
oleh publik dan siapa saja dapat ikut serta untuk mengembangk-
an blockchain, Private Blockchain yang hanya bisa digunakan oleh
organisasi tertentu, consortium blockchain yang dikembangkan oleh
suatu kelompok secara bersama untuk kepentingan tertentu, seder-
hananya blockchain konsorsium merupakan blockchain privat yang
diberdayakan oleh lebih dari satu kelompok.

\subsubsection{Unreal Engine 5}

Unreal Engine (UE) adalah Game Engine grafis komputer 3D yang dikembangkan oleh Epic Games, pertama kali dipamerkan dalam game penembak orang pertama tahun 1998 Unreal. 
Awalnya dikembangkan untuk penembak orang pertama PC, sejak itu telah digunakan dalam berbagai genre permainan dan telah diadopsi oleh industri lain, terutama industri film dan televisi. Ditulis dalam C++, 
Unreal Engine memiliki fitur portabilitas tingkat tinggi, mendukung berbagai platform desktop, seluler, konsol, dan virtual reality (VR).
Generasi terbaru, Unreal Engine 5, diluncurkan pada April 2022. Seperti pendahulunya yang dirilis pada Maret 2014, kode sumbernya tersedia di GitHub setelah mendaftarkan akun, dan penggunaan komersial diberikan 
berdasarkan model royalti. Epic membebaskan margin royalti mereka untuk game sampai pengembang memperoleh pendapatan US\$1 juta dan biaya tersebut dibebaskan jika pengembang menerbitkan di Epic Games Store. Epic telah memasukkan fitur dari perusahaan yang diakuisisi seperti Quixel di mesin, yang dipandang terbantu oleh pendapatan Fortnite.
Unreal Engine 5 terungkap pada 13 Mei 2020, mendukung semua sistem yang ada termasuk konsol generasi berikutnya PlayStation 5 dan Xbox Series X/S.\citep{StattEpicAnnounce} Pengerjaan mesin dimulai sekitar dua 
tahun sebelum diumumkan.\citep{DeanTakahashi} UE5 dirilis dalam early access pada 26 Mei 2021,\citep{EddieMakuch} dan secara resmi diluncurkan untuk pengembang pada 5 April 2022.\citep{UE5Launch}

Salah satu fitur utamanya adalah Nanite, mesin yang memungkinkan bahan sumber fotografi dengan detail tinggi diimpor ke dalam game.\citep{ValentineRebekah} Teknologi geometri tervirtualisasi Nanite memungkinkan Epic memanfaatkan akuisisi Quixel sebelumnya, perpustakaan fotogrametri terbesar di dunia pada 2019. Tujuan Unreal Engine 5 adalah untuk memudahkan pengembang 
membuat dunia game yang mendetail tanpa harus menghabiskan banyak waktu untuk membuat aset detail baru.\citep{DeanTakahashi} Nanite dapat mengimpor hampir semua representasi objek dan lingkungan tiga dimensi yang sudah ada sebelumnya, termasuk model ZBrush dan CAD, memungkinkan penggunaan aset berkualitas film.\citep{AndrewTarantola} Nanite secara otomatis menangani tingkat detail (LODs) 
dari objek yang diimpor ini sesuai dengan platform target dan jarak menggambar, tugas yang harus dilakukan oleh seorang seniman jika tidak.\citep{KyleOrland} Lumen adalah komponen lain yang digambarkan sebagai "solusi iluminasi global yang sepenuhnya dinamis yang segera bereaksi terhadap pemandangan dan perubahan cahaya". Lumen menghilangkan kebutuhan 
seniman dan pengembang untuk membuat peta cahaya untuk adegan tertentu, tetapi sebaliknya menghitung pantulan cahaya dan bayangan dengan cepat, sehingga memungkinkan perilaku sumber cahaya waktu nyata.\citep{KyleOrland} Virtual Shadow Maps adalah komponen lain yang ditambahkan di Unreal Engine 5 yang digambarkan sebagai "metode pemetaan bayangan baru yang digunakan 
untuk memberikan bayangan resolusi tinggi yang konsisten yang bekerja dengan aset berkualitas film dan dunia terbuka yang besar dan menyala secara dinamis"..\citep{VirtualShadowMaps} Peta Bayangan Virtual berbeda dari implementasi peta bayangan pada umumnya dalam resolusi yang sangat tinggi, bayangan yang lebih detail, dan kurangnya bayangan yang muncul dan keluar yang 
dapat ditemukan dalam teknik peta bayangan yang lebih umum karena kaskade bayangan.[121] Komponen tambahan termasuk Niagara untuk dinamika fluida dan partikel dan Chaos untuk mesin fisika.\citep{DeanTakahashi}
