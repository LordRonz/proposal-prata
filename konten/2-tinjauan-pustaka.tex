\section{TINJAUAN PUSTAKA}

% Ubah konten-konten berikut sesuai dengan isi dari tinjauan pustaka

\subsection{Jaringan Terdistribusi}

Jaringan Terdistribusi adalah sebuah jaringan dimana setiap
node dalam cluster tersebut saling terhubung satu sama lain. Ja-
ringan desentralisasi mendistribusikan data, informasi, dan memp-
roses beban kerja di seluruh komputer yang berpartisipasi dalam
jaringan. Hal ini memungkinkan untuk toleransi kesalahan yang
lebih besar sebagai komponen kunci dari jaringan di distribusikan
di beberapa mesin, jika salah satu jaringan (hardware) down, maka
jaringan secara keseluruhan terus berfungsi.
Jaringan terdesentralisasi memiliki keuntungan lebih aman da-
ripada jaringan terpusat, tidak ada satu server pusat yang ditargetk-
an oleh penyerang. Jika seseorang ingin menyerang jaringan, mereka
perlu untuk menyerang beberapa node dalam jaringan. Karena ti-
dak ada kekuasaan dalam jaringan, pengguna bisa lebih memperca-
yai sistem, namun juga karena tidak ada kekuasaan dalam jaringan,
apabila ada masalah dalam jaringan tidak ada otoritas pusat yang
membantu menyelesaikan masalah hal ini yang membedakan anta-
ra jaringan terpusat dan jaringan terdistribusi.

\subsection{Blockchain}

Blockchain adalah database terdistribusi yang digunakan un-
tuk menangani record data yang terus bertambah, record data ini
disebut dengan block. Setiap block memiliki penanda waktu dan ko-
de unik yang terhubung dengan block sebelumnya, sehingga masing
masing block tersebut saling terhubung satu sama lainnya dan ti-
dak bisa untuk diubah. Blockchain biasanya dikelola oleh jaringan
peer-to-peer yang secara kolektif mengikuti protokol untuk memva-
lidasi block baru. Jika terdapat perintah penambahan block baru,
maka setiap node pada jaringan peer-to-peer tersebut akan terlebih
dahulu memvalidasi block dan kemudian seluruh node akan mem-
perbarui record data miliknya [18]. Blockchain memiliki 3 type [19]
yaitu Public Blockchain yang dikembangkan secara bersama-sama
oleh publik dan siapa saja dapat ikut serta untuk mengembangk-
an blockchain, Private Blockchain yang hanya bisa digunakan oleh
organisasi tertentu, consortium blockchain yang dikembangkan oleh
suatu kelompok secara bersama untuk kepentingan tertentu, seder-
hananya blockchain konsorsium merupakan blockchain privat yang
diberdayakan oleh lebih dari satu kelompok.
